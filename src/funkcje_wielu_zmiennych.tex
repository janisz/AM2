
\section{Funkcje wielu zmiennych}
\subsection{Granica funkcji}
\begin{df}[Granicy wg Heinego]
Mówimy, że $g$ jest granicą funkcji $f$ w punkcie $a$
$$\Leftrightarrow ~~\forall \{\mathbf{x}_n \subset D \smallsetminus \{a\} \} \quad
[\lim\limits_{n\to\infty}\mathbf{x}_n = a \Rightarrow \lim\limits_{n\to\infty}f(\mathbf{x}_n)=g]$$
\end{df}
\begin{df}[Granicy wg Cauchego]
Mówimy, że $g$ jest granicą funkcji $f$ w punkcie $a$
$$\Leftrightarrow \forall\varepsilon >0 ~~\exists \delta >0 ~~\forall x\in D \quad 0<||\mathbf{x} - a|| <\delta
\Rightarrow |f(\mathbf{x})-g| < \varepsilon$$
\end{df}
\begin{df}[Granicy niewłaściwej wg Heinego]
$$\lim\limits_{x\to a} = \pm \infty~~\Leftrightarrow ~~\forall \{\mathbf{x}_n \subset D \smallsetminus \{a\} \} \quad
[\lim\limits_{n\to\infty}\mathbf{x}_n = a \Rightarrow \lim\limits_{n\to\infty}f(\mathbf{x}_n)=\pm\infty]$$
\end{df}
\begin{df}[Granicy niewłaściwej wg Cauchego]
$$\lim\limits_{x\to a} = +\infty~~\Leftrightarrow \forall M\in\mathbb{R}~~\exists \delta >0 ~~\forall x\in D \quad
0<||\mathbf{x} - a|| <\delta \Rightarrow f(x) > M$$
$$\lim\limits_{x\to a} = -\infty~~\Leftrightarrow \forall M\in\mathbb{R}~~\exists \delta >0 ~~\forall x\in D \quad
0<||\mathbf{x} - a|| <\delta \Rightarrow f(x) < M$$
\end{df}

\subsection{Ciągłość funkcji}
\begin{df}
Funkcja $f$ jest ciągła w punkcie $a\in D \Leftrightarrow [\exists \lim\limits_{x\to a}f(x) \text{ i }
\lim\limits_{x\to a}f(x) = f(a)]$
\end{df}
\begin{df}
Funkcja $f$ jest ciągła $\Leftrightarrow$ jest ciągła w każdym punkcie $a\in D$
\end{df}

\begin{tw}[O ciągłości działań arytmetycznych]~\\
$[f,g: D\to \mathbb{R}, D\subset\mathbb{R}^k, a\in D; f,g \text{ są ciągłe w} a, x\in\mathbb{R}] \Rightarrow$
ciągłe w punkcie są następujące funkcje

\begin{itemize}
	\item $|f| $
	\item $\lambda f $
	\item $f \pm g $
	\item $fg$
	\item $\frac{f}{g} \text{ jeśli tylko } g(a) \neq 0$
\end{itemize}

\end{tw}
\begin{przyklad}
$ f:\mathbb{R}^2 \to\mathbb{R}, f(xy) \begin{cases}
	\frac{xy}{x^2+y^2} & \text{ dla } (x,y) \neq (0,0)\\
	0 & \text{ dla } (x,y) = (0,0)
\end{cases}$\\

\begin{itemize}
	\item $f$ jest ciągła na $\mathbb{R}^2 \smallsetminus \{(0,0)\}$ jako iloraz funkcji ciągłych (wielomianów)
	\item $f$ nie jest ciągła w punkcie $(0,0)$, bo nie istnieje granica $\lim\limits_{(x,y)\to(0,0)}f(x,y)$
\end{itemize}
\end{przyklad}

\subsection{Pochodne i różniczkowalność funkcji wielu zmiennych}
\begin{df}[Pochodnej cząstkowej]~\\
Jeśli istnieje skończona granica $\lim\limits_{h\to 0} \frac{f(x_{10}+h, x_{20}, \dots, x_{k0})-f(x_{10}, x_{20},
\dots, x_{k0})}{h}$ to nazywamy ją pochodna cząstkową funkcji $f$ względem zmiennej $x_1$ w punkcie $x_0 = (x_{10},
x_{20}, \dots, x_{k0})$ i oznaczamy $f'_{x_1}(x_0)$
\end{df}
\begin{df}[Pochodnej kierunkowej]~\\
Pochodną kierunkowa funkcji $f$ w punkcie $x_0 = (x_{10}, x_{20}, \dots, x_{k0})$ w kierunku wektora  $v = (v_{1},
v_{2}, \dots, v_{k})$ o długości 1 (to znaczy $||v|| = 1$) nazywamy granicę $\lim\limits_{t\to  0}\frac{f(x_0+tv) -
f(x_0)}{t}$ jeśli istnieje i jest skończona. Oznaczamy ją $f'_v(x_0)$
\end{df}
\begin{df}[Różniczkowalności funkcji]~\\
Niech $G\subset \mathbb{R}^k$ będzie obszarem i $f:G\to \mathbb{R}$. Mówimy że $f$ jest różniczkowalna w punkcie
$$x_0\in G \Leftrightarrow \exists A=(a_1, a_2, \dots, a_k) \in \mathbb{R}^k \quad \lim\limits
_{h\to 0}\frac{f(x_0+h)-f(x_0)-Ah}{||h||} = 0$$
\end{df}

%dodać definicję gradientu

\begin{tw}[Warunek konieczny różniczkowalności]~\\
$f:G\to\mathbb{R}, ~ G\subset\mathbb{R}^m$ to obszar, $f$ jest różniczkowalna w punkcie $x_0 \Rightarrow f$ jest
ciągła w punkcie $x_0$
\begin{proof}
$f$ różniczkowalna w punkcie $x_0 \Rightarrow \lim\limits_{h\to 0}\frac{f(x_0+h)-f(x_0)-Ah}{||h||}=0$ dla pewnego
$A=(a_1,a_2, \dots, a_m) \in \mathbb{R}^m$\\
Oznaczmy $\eta_{x_0}(h) = \frac{f(x_0+h)-f(x_0)-Ah}{||h||}$, wówczas mamy $\lim\limits_{n\to 0}\eta_{x_0}(h) = 0$\\
$f(x_0+h)=\eta_{x_0}(h) = \eta_x(h)||h|| + f(x_0) + Ah$\\
$\lim\limits_{h\to 0}f(x_0+h) = \lim\limits_{h\to 0}[\eta_{x_0}(h)] = \lim\limits_{h\to
0}[\underbrace{\eta_x(h)||h||}_{0} + f(x_0) + \underbrace{a_1h_1 + a_2h_2 + \dots + a_mh_m}_{0}] = f(x_0) \Rightarrow
f$ jest ciągła w punkcie $x_0$
\end{proof}
\end{tw}

\begin{tw}[warunek dostateczny różniczkowalności]~\\
$f: G\to\mathbb{R}, ~ G\subset \mathbb{R}^m$ to obszar\\
Istnieją pochodne cząstkowe $f'_{x_1}(x)\dots f'_{x_m}(x)$ w pewnym otoczeniu punktu $x_0$ i są one ciągłe w
punkcie $x_0 \Rightarrow f$ jest różniczkowalna w punkcie $x_0$
\end{tw}

\subsection{Pochodne cząstkowe wyższych rzędów}
\begin{df}
Niech $f: G\to\mathbb{R}$, gdzie $G\subset \mathbb{R}^m$ to obszar, ma pochodne cząstkowe $f'_{x_i}, i=1,\dots , m$\\
Jeśli funkcja $f'_{x_i}$ ma pochodną cząstkową po zmiennej $x_j$ to nazywamy ją pochodną czastkową drugiego
rzędu i oznaczamy $f''_{x_ix_j} = (f'_{x_i})'_{x_j}$
\end{df}

\begin{tw}[Schwarza]~\\
Niech $f: G\to\mathbb{R}$, gdzie $G\subset \mathbb{R}^2$ to obszar.\\
Pochodne mieszane $f''_{xy}$ i $f''_{yx}$ istnieją w pewnym otoczeniu punktu $(x_0, y_0)\in G$ i są ciągłe w tym
punkcie $\Rightarrow f''_{xy} = f''_{yx}$
\end{tw}

\begin{tw}[Wzór Taylora dla funkcje 2 zmiennych]~\\
Jeśli $f$ jest funkcją klasy $C^n$ w pewnym obszarze zawierającym odcinek $\overline{x_0x}$, to wewnątrz  tego
odcinka znajduje się punkt $\mathbf{c} = (c_1,c_2)$ taki że\\
$
f(x) = f(x_0) + \frac{1}{1!}\left[\frac{df}{dx}(x_0)(x-x_0) + \frac{df}{dy}(x_0)(y-y_0)\right] +
				\frac{1}{2!}\left[\frac{d^2f}{dx^2}(x_0)(x-x_0)^2 +
2\frac{d^2f}{dxdy}(x_0)(x-x_0)(y-y_0) + \frac{d^2f}{dy^2}(x_0)(y-y_0)^2\right]
				+ \dots + \frac{1}{(n-1)!}\sum\limits_{i=0}^{n-1}{n \choose
i}\frac{d^nf}{dx^idy^j}(x_0)(x-x_0)^i(y-y_0)^{n-i-1} + R_n(x_0) \text{, gdzie } R_n(x_0) =
\frac{1}{n!}\sum\limits_{i=0}^n {n \choose i}\frac{d^n}{dx^idY^{n-1}}(c)(x-x_0)^i(y-y_0)^{n-1}
$
\end{tw}

\subsection{Ekstrema funkcji wielu zmiennych}
\begin{df}
Mówimy, że funkcja $f$ ma w punkcie $x_0\in G$:\\
\begin{itemize}
	\item maksimum lokalne jeśli $\exists r>0 ~\forall x\in K(x_0,r) \quad f(x) \leqslant f(x_0)$
	\item maksimum lokalne właściwe jeśli $\exists r>0 ~\forall x\in K(x_0,r), x \neq x_0  \quad f(x) < f(x_0)$
	\item minimum lokalne jeśli $\exists r>0 ~\forall x\in K(x_0,r) \quad f(x) \geqslant f(x_0)$
	\item minimum lokalne właściwe jeśli $\exists r>0 ~\forall x\in K(x_0,r), x \neq x_0  \quad f(x) > f(x_0)$
\end{itemize}
\end{df}

\begin{tw}[Warunek konieczny istnienia ekstremum funkcji wielu zmiennych]~\\
Jeśli istnieją pochodne cząstkowe $f'_{x_1}(x_0), f'_{x_2}(x_0), \dots, f'_{x_k}(x_0)$ i funkcja ma w punkcie $x_0$
ekstremum lokalne to $f'_{x_1}(x_0) = f'_{x_2}(x_0) = \dots = f'_{x_k}(x_0) = 0$
\end{tw}

\begin{tw}[Warunek dostateczny istnienia ekstremum dla funkcji 2 zmiennych]~\\
Niech $G\subset \mathbb{R}^2$ będzie obszarem, $f\in C^2(G)$ i $(x_0,y_0) \in G$. Jeśli $f'_x(x_0, y_0) =
f'_y(x_0,y_0) = 0$ i $W(x_0,y_0) =
\begin{vmatrix}
	f''_{xx}(x_0,y_0) & f''_{xy}(x_0,y_0)\\
	f''_{yx}(x_0,y_0) & f''_{yy}(x_0,y_0)\\
\end{vmatrix} > 0$ to w punkcie $(x_0,y_0)$ jest ekstremum lokalne właściwe. Ponadto jeśli $f''_{xx}(x_0,y_0)>0$ to
jest to minimum lokalne, a jeśli $f''_{xx}(x_0,y_0)<0$ to maksimum lokalne
\end{tw}

\begin{df}
Formę kwadratową, a także odpowiadającą jej macierz $A$ nazywamy
\begin{itemize}
	\item dodatnio określoną $\Leftrightarrow \forall \mathbf{x}\neq 0 \quad \mathbf{x}A\mathbf{x}^T > 0$
	\item ujemnie określoną $\Leftrightarrow \forall \mathbf{x}\neq 0 \quad \mathbf{x}A\mathbf{x}^T < 0$
	\item nieujemnie określoną $\Leftrightarrow \forall \mathbf{x}\neq 0 \quad \mathbf{x}A\mathbf{x}^T \geqslant
0$
	\item niedodatnio określoną $\Leftrightarrow \forall \mathbf{x}\neq 0 \quad \mathbf{x}A\mathbf{x}^T \leqslant
0$
	\item nieokreśloną $\Leftrightarrow \forall \mathbf{x,y} \quad \mathbf{x}A\mathbf{x}^T < 0,~
\mathbf{y}A\mathbf{y}^T > 0$
\end{itemize}
\end{df}

\begin{tw}[Kryterium Sylwestera]~\\
Macierz symetryczna $k\times k$:
\begin{itemize}
	\item jest dodatnio określona $\Leftrightarrow \forall i=1,\dots ,k ~ \det A^{(i)}>0$
	\item jest ujemnie określona $\Leftrightarrow \forall i=1,\dots ,k ~ (-1)^i\det A^{(i)}>0$
\end{itemize}
gdzie $A^{(i)}$ to macierz powstała z $A$ przez skreślenie kolumn i wierszy o numerach większych niż $i$
\end{tw}

\begin{tw}[warunek wystarczający istnienia ekstremum dla funkcji wielu zmiennych]~\\
Niech $G\subset \mathbb{R}^k$ będzie obszarem i niech $f\in C^2(G)$. Jeśli $\mathbf{x_0}\in G, f'x_0(\mathbf{x_0}) =
f'x_1(\mathbf{x_0}) = \dots = f'x_n(\mathbf{x_0})$ to
\begin{itemize}
	\item Jeśli macierz $H(\mathbf{x_0})$ jest dodatnio określona, to $f$ ma minimum lokalne w $\mathbf{x_0}$
	\item Jeśli macierz $H(\mathbf{x_0})$ jest ujemnie określona, to $f$ ma maksimum lokalne w $\mathbf{x_0}$
	\item Jeśli macierz $H(\mathbf{x_0})$ jest nieokreślona, to $f$ nie posiada ekstremum lokalnego w
$\mathbf{x_0}$
\end{itemize}
\end{tw}

\begin{tw}[O funkcji uwikłanej]~\\
Załóżmy że $F$ ma ciągłe pochodne cząstkowe w pewnym otoczeniu punktu $(x_0, y_0)$ oraz $F(x_0,y_0) = 0,~
F'_y(x_0,y_0) \neq 0$ to istnieje otoczenie $U_{x_0} \ni x_0$ i otoczenie $V_{y_0} \ni y_0$ oraz jedyna funkcja $y:
U_{x_0} \to V_{y_0}$ klasy $C^1$ taka że $\forall x\in U_{x_0} \quad F(x, y(x)) = 0$. Ponadto $y$ ma ciągłą
pochodną w $U_{x_0}$ i $y'(x) = \frac{-F_x(x,y)}{F'_y(x,y)}$, dla $x\in U_{x_0}$
\end{tw}

\subsection{Całka Riemanna w $\mathbb{R}^n$}
\begin{df}
Ciąg podziałów $(\pi(n))$ nazywamy normalnym, jeśli $\delta (\pi (n)) \xrightarrow{n\to\infty}0$
\end{df}
\begin{df}
Jeśli istnieje stała $\delta\in\mathbb{R}$ taka, że dla dowolnego podziału $\pi_{n}$ normalnego prostokąta $P$ i
dla dowolnego wartościowania $w_{n}$ mamy $\delta=\lim\limits_{n\to\infty}\delta_{n}(\pi_{n},w_{n}) $, to mówimy, że
$f$ jest całkowalna w sensie Riemanna na $P$ i zapisujemy $\delta=\iint f(x,y)dxdy$, oraz $\delta$ nazywamy całką
podwójną  funkcji $f$ na prostokącie $P$.
\end{df}
\begin{tw}
Dla każdego ciągu podziałów $\pi_{n}$ istnieją granice $\lim\limits_{n\to\infty}s_{n}(\pi_{n})$ i
$\lim\limits_{n\to\infty}S_{n}(\pi_{n})$ i granice te zależą od wyboru ciągu podziałów:
\begin{itemize}
	\item$s:=\lim\limits_{n\to\infty}s_{n}(\pi_{n})$ (całka podwójna górna),
	\item$S:=\lim\limits_{n\to\infty}S_{n}(\pi_{n})$(całka podwójna dolna)
\end{itemize}
\end{tw}