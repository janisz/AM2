\documentclass[12pt,a4paper]{article}
\usepackage[centertags]{amsmath}
\usepackage[polish]{babel}
\usepackage[utf8]{inputenc}
\usepackage[T1]{fontenc}
\usepackage[a4paper]{geometry}
\usepackage{amsfonts}
\usepackage{fullpage}
%\usepackage{epsfig}
\usepackage{amsthm}
\usepackage{pdflscape}
\usepackage{float}
%\usepackage{newlfont}
\usepackage{amsmath} %pakiet potrzebny do \eqref
\usepackage{amssymb} %pakiet potrzebny do \blacksquare i \mathbb{•}
\usepackage{hyperref}

\newtheorem{lemat}{Lemat}
\newtheorem{tw}{Twierdzenie}
\newtheorem{przyklad}{Przykład}
\newtheorem{wn}{Wniosek}
\newtheorem{zad}{Zadanie}
\theoremstyle{definition}
\newtheorem{df}{Definicja}
\usepackage[polish]{babel}
\usepackage[utf8]{inputenc}

 \hypersetup{colorlinks=false}

\pagestyle{plain}
\begin{document}
\title{ Analiza matematyczna 2}
\date{\today}
\maketitle
\tableofcontents
\pagebreak

\section{Całka Riemanna}
\begin{df}[Ciąg normalny podziałów]
Ciąg podziałów $(\pi(n)$ nazywamy normalnym, jeśli $\delta(\pi(n))\xrightarrow{n\to\infty}0$
\end{df}
\begin{df}[Sumy całkowe]~\\
\begin{array}{l l}
\mbox{Dolna suma całkowa:}& $s_n = s_n(\pi_n) = \sum\limits_{i=1}^{k_n}m_i^{(n)}(x_i^{(n)}-x_{i-1}^{(n)}})$\\
\mbox{Górna suma całkowa:}& $S_n = S_n(\pi_n) = \sum\limits_{i=1}^{k_n}M_i^{(n)}(x_i^{(n)}-x_{i-1}^{(n)}})$\\
\mbox{Suma całkowa Riemanna:}& $\sigma_n = \sigma_n(\pi_n) = \sum\limits_{i=1}^{k_n}f(\xi_i^{(n)})(x_i^{(n)}-x_{i-1}^{(n)}})$
\end{array}
\end{df}
\begin{df}[Funkcja całkowalna w sensie Riemanna]
Funkcja f jest całkowalna w sensie Riemanna na $[a,b] \Leftrightarrow$ instnieje $\sigma\in\mathbb{R}$ taka,że dla dowolnego normalnego ciągu podziałów $(\pi_n)$ oraz dla dowolnego wartościowania tego ciągu $(\omega_n)$ mamy $\lim\limits_{n\to\infty}\sigma_n(\pi_n,\omega_0) = \lim\limits_{n\to\infty}\sum\limits_{i=1}^{k_n}f(\xi_i^{(n)})(x_i^{(n)}-x_{i-1}^{(n)}})$, to $\sigma$ nazywamy całką Riemanna funkcji f na $<a,b>$ i oznaczamy $\int\limits_a^b(f(x)dx$
\end{df}
\begin{tw}
$f: [a,b] \rightarrow \mathbb{R}$ (ograniczona) jest całkowalna w sensie Riemanna na $[a,b] \Leftrightarrow s=S$
\end{tw}
\begin{tw}
Każda funkcja monoticzna i ograniczona na $[a,b]$ jest całkowalna w sensie Riemanna na $[a,b]$
\end{tw}
\begin{tw}
Każda funkcja ciągła na $[a,b]$ jest całkowalna w sensie Riemanna na $[a,b]$
\end{tw}

\subsection{Własności całki Riemanna}

\begin{enumerate}
\item $f,g: [a,b] \to \mathbb{R} {x\in[a,b]: f(x) \neq g(x)}$ jest zbiorem skończonym $\Rightarrow$ f jest całkowalna na $[a,b]\Leftrightarrow$ g jest całkowalna na $[a,b]$. W przypadku całkowalności $\int\limits_a^bf(x)dx=\int\limits_a^bg(x)dx$
\item Liniowość
\item $f: [a,b]\to\mathbb{R}$ całkowalna na $[a,b] \Rightarrow$ f jest całkowalna na kazdym podprzedziale $[a,b]$
\item $f: [a,b]\to\mathbb{R}$, f jest całkowalna na $[a,c]$ oraz $[c,b] \Rightarrow$ f jest całkowalna na $[a,b]$ oraz $\int\limits_a^bf(x)dx = \int\limits_a^cf(x)dx + \int\limits_c^bf(x)dx$
\item $f: [a,b] \to \mathbb{R}$ całkowalna na $[a,b]$, $\forall x\in[a,b] f(x) \geq 0 \Rightarrow \int\limits_a^bf(x)dx \geq 0$
\item $f,g: [a,b] \to \mathbb{R}$ całkowalne na $[a,b]$, $\forall x\in[a,b] f(x) \leq g(x) \Rightarrow \int\limits_a^bf(x)dx \leq \int\limits_a^bg(x)dx$
\item $f: [a,b] \to \mathbb{R}$ całkowalna na $[a,b] \Rightarrow$ $|f|$ też jest całkowalna na $[a,b]$ oraz $|\int\limits_a^bf(x)dx| \leq \int\limits_a^b|f(x)|dx$
\item $f: [a,b] \to \mathbb{R}$ całkowalna na $[a,b] \Rightarrow \int\limits_a^bf(x)dx \leq \sup\limits_{x\in[a,b]}f(x)\cdot(b-a)$
\item Podstawowy wzór rachunku całkowego\\
$f: [a,b] \to \mathbb{R}$ funkcja ciągła, $\phi(x)$ --- dowolna funkcja pierwotna dla $f(x)$, to znaczy $\phi'(x)=f(x) \Rightarrow \int\limits_a^bf(x)dx = \phi(b)-\phi(a)$
\item Całkowanie przez podstawienie\\
$
f: [a,b]\to\mathbb{R}$, funkcja ciągła, $g: [a,b]\to\mathbb{R},~ g\in C^1([a,b])\\
\alpha = g(a),~ \beta =g(b)\\
\int\limits_a^bf(g(x))g'(x)dx = \int\limits_\alpha^\beta f(t)dt, \mbox{ gdzie }t = g(x)
$
\begin{proof}~\\
	Niech $\phi(x)$ będzie funkcją pierwotną dla $f(x)$, to znaczy
	 $\phi'(x) = f(x), \phi=\int\limits_a^bf(t)dt = \phi(\beta)-\phi(\alpha)$\\
$[\phi(g(x))]' = \phi'(g(x))g'(x) = f(g(x))g'(x) \Rightarrow \phi(g(x)) \mbox{ to funkcja pierwotna funkcji } f(g(x))g'(x)$\\
$L = \int\limits_a^bf(g(x))g'(x)dx = \phi(g(b))-\phi(g(a)) = \phi(\beta)-\phi(\alpha)\\
L = P$
\end{proof}
\item Całkowanie przez cześci\\
$
u,v: [a,b]\to\mathbb{R} \mbox{ --- funkcje klasy } C'\\
\int\limits_a^bu(x)v'(x)dx = [u(x)v(x)]\limits_a^b - \int\limits_a^bu'(x)v(x)dx
$
\begin{proof}~\\
$
	[u(x)v(x)]' = u'(x)v(x)+u(x)v'(x)\\
	\int\limits_a^b[u(x)v(x)]'dx = \int\limits_a^b(u'(x)v(x)+u(x)v'(x))dx = 
	\int\limits_a^b u'(x)v(x)dx + \int\limits_a^bu(x)v'(x)dx \quad(*)\\
	u(x)v(x) \mbox{ to funkcja pierwotna } [u(x)v(x)]'\\
	\int\limits_a^b[u(x)v(x)]'dx u(b)v(b) - u(a)v(a) = [u(x)v(x)]\limits_a^b \quad(**)\\
	(*):(**) \Rightarrow [u(x)v(x)]\limits_a^b = \int\limits_a^bu'(x)v(x)dx + \int\limits_a^b u(x)v'(x)dx\\
	\int\limits_a^bu(x)v(x)dx = [u(x)v(x)]_a^b-\int\limits_a^bu'(x)v(x)dx	
$
\end{proof}
\item Twierdzenie o wartości średniej rachunku całkowego
$
f,g: [a,b]\to\mathbb{R} \mbox{ --- funkcje ciągłe, g jest nieujemna (niedodatnia)}\\
 \exists \xi\in [a,b] \int\limits_a^bf(x)g(x)dx = f(\xi)\int\limits_a^bg(x)dx
$
\end{enumerate}

\subsection{Całki niewłaściwe}

\begin{df}
$f: [a, \infty) \to \mathbb{R} \mbox{ całkowalna na } [\alpha, \beta] \forall \beta > \alpha$ oraz istnieje granica $\lim\limits_{\beta\to\infty} \int\limits_\alpha^\beta f(x)dx \Rightarrow$ granicę tę nazywamy całka niewłaściwą pierwszego rodzaju i oznaczamy $\int\limits_\alpha^\beta f(x)dx := \lim\limits_{\beta\to\infty} \int\limits_\alpha^\beta f(x)dx$\\
Ponadto jeśli granica ta istnieje i jest skończona to całkę niewłaściwą nazywamy zbieżną. Natomiast w pozostałych przypadkach całkę niewłaściwą nazywamy rozbieżną.
\end{df}

\begin{df}
$f: [a, \infty) \to \mathbb{R} \mbox{ całkowalna na } [\alpha, \beta] \quad \forall a < \beta < b$ oraz istnieje granica $\lim\limits_{\beta\to b^-} \int\limits_a^\beta f(x)dx \Rightarrow$ granicę tę nazywamy całka niewłaściwą drugiego rodzaju i oznaczamy $\int\limits_a^b f(x)dx := \lim\limits_{\beta\to b^-} \int\limits_\alpha^\beta f(x)dx$\\
Ponadto jeśli granica ta istnieje i jest skończona to całkę niewłaściwą nazywamy zbieżną. Natomiast w pozostałych przypadkach całkę niewłaściwą nazywamy rozbieżną.
\end{df}

\subsection{Kryteria zbieżności}

\begin{tw}[Kryterium porównawcze]
$f,g: [a,b) \to \mathbb{R} \mbox{ całkowalna na } [\alpha, \beta] \quad \forall a < \beta < b$ oraz
$\forall x\in [a,b) \quad 0<leq f(x) \leq g(x)$ Wtedy:
\begin{itemize}
\item $\int\limits_a^bg(x)dx$ jest zbieżna $\Rightarrow \int\limits_a^bg(x)dx$ też jest zbieżna
\item $\int\limits_a^bf(x)dx$ jest rozbieżna $\Rightarrow \int\limits_a^bg(x)dx$ też jest rozbieżna
\end{itemize}
\end{tw}

\begin{tw}
$f: [a,b) \to \mathbb{R} \mbox{ całkowalna na } [\alpha, \beta] \quad \forall a < \beta < b$ oraz
$\int\limits_a^b \f(x)|dx$ jest zbieżna, to $\int\limits_a^b f(x)dx$ też jest zbieżna. Mówimy wtedy, że jest zbieżna bezwzględnie.
\end{tw} 

\subsection{Zastosowania geometryczne całki Riemanna}

\begin{tw}[Pole zbioru płaskiego]
$f: [a,b] \to \mathbb{R}$ funkcja nieujemna i ciągła, $D = \{(x,y) \in \mathbb{R}^2: x\in [a,b[, y\in [0, f(x)]$\\
pole $D = |D| = \int\limits_a^b f(x)dx$
\end{tw}

\begin{tw}[Długość łuku]
Niech $f: [a,b] \to \mathbb{R}$ - funkcja klasy $C^1$. Wówczs długość łuku opisanego równaniem $y=f(x), x\in [a,b]$ dana jest wzorem $L = \int\limits_a^b \sqrt{1 + \left[f'(x)\right]^2}$
\end{tw}

\begin{tw}[Objętość bryły]
Niech $f: [a,b] \to \mathbb{R}$ - funkcja klasy $C^1$ oraz $V$ oznacza bryłę powstałą poprzez obrót
 dookoła osi OX krzywej $y - f(x), x\in [a,b]$. Wówczs objętość $V$ dana jest wzorem $V = \pi\int\limits_a^b f^2(x)dx$
\end{tw}


\section{Szeregi liczbowe}

\begin{df}[sumy szeregu liczbowego]~\\
Jeśli istnieje skończona lub nie $\lim\limits_{n\to\infty}S_n$, to nazywamy ją sumą szeregu $\sum\limits_{n=1}^\infty a_n$ i zapisujemy $\sum\limits_{n=1}^\infty a_n=\lim\limits_{n\to\infty}S_n$.
Jeśli $\lim\limits_{n\to\infty}S_n$ jest skończona, to szereg $\sum\limits_{n=1}^\infty a_n$ nazywamy zbieznym; w pozostałych przypadkach (to znaczy gdy granica jest nieskończona lub nie istnieje) szereg ten nazywamy rozbieżnym
\end{df}

\begin{tw}[Warunek konieczny zbieżności szeregu]~\\
$\sum\limits_{n=1}^\infty an$ jest zbieżny $\Rightarrow \lim\limits_{n\to\infty}a_n = 0$
\begin{proof}
Zakładamy, że $\sum\limits_{n=1}^\infty an$ jest zbieżny $\Rightarrow (S_n)$ jest zbieżny; oznaczamy 
$S_n \xrightarrow{n\to\infty} S$\\
$a_n = S_n - S_{n-1} \xrightarrow{n\to\infty} S - S = 0$ 
\end{proof}
\begin{przyklad}[Powyższy warunek nie jest warunkiem dostatecznym]~\\
$\sum\limits_{n=1}^\infty \frac{1}{n}$ nie jest zbiezny, mimo że $\lim\limits_{n\to\infty} \frac{1}{n} = 0$
\end{przyklad}
\end{tw}

\begin{tw}
Szereg $\sum\limits_{n=1}^\infty a_n$ jest zbieżny $\Leftrightarrow$ spełnia warunek Cauch'ego, to znaczy
$$
\forall \varepsilon > 0 ~~\exists N ~~\forall m > n > N \quad |a_{n+1} + a_{n+2} + \dots + a_m| < \varepsilon
$$
\end{tw}

\begin{tw}[O mnożeniu szeregu przez stałą]~\\
$\sum\limits_{n=1}^\infty a_n$ jest zbieżny i $\lambda\in\mathbb{R}$ wówczas szereg $\sum\limits_{n=1}^\infty \lambda a_n$ jest zbieżny i $\sum\limits_{n=1}^\infty \lambda a_n = \lambda \sum\limits_{n=1}^\infty a_n$
\end{tw}
\begin{tw}[O dodawaniu i odejmowaniu szeregów]~\\
$\sum\limits_{n=1}^\infty a_n, \sum\limits_{n=1}^\infty b_n$ są zbieżne wówczas szereg $\sum\limits_{n=1}^\infty a_n \pm \sum\limits_{n=1}^\infty b_n$ jest zbieżny i $\sum\limits_{n=1}^\infty a_n \pm \sum\limits_{n=1}^\infty b_n = \sum\limits_{n=1}^\infty (a_n \pm b_n)$
\end{tw}

\subsection{Kryteria zbiezności szeregów}

\begin{tw}[Kryterium porównawcze]~\\
$\forall n\in\mathbb{N} \quad 0 \leqslant a_n \leqslant b_n$, wówczas
$\lim\limits_{n \to \infty} \frac{a_{n+1}}{a_n} = g$ wówczas:
\begin{itemize}
	\item $\sum\limits_{n=1}^\infty b_n$ jest zbieżny $\Rightarrow \sum\limits_{n=1}^\infty a_n$ jest zbieżny
	\item $\sum\limits_{n=1}^\infty a_n$ jest rozbieżny $\Rightarrow \sum\limits_{n=1}^\infty b_n$ jest rozbieżny
\end{itemize}
\end{tw}

\begin{tw}[Kryterium d'Alemberta]~\\
Niech $\forall n\in\mathbb{N} \quad a_n > 0$ i istnieje granica 
$\lim\limits_{n\to\infty} \frac{a_{n+1}}{a_n} = g$, wówczas
\begin{itemize}
	\item $g < 1 \Rightarrow \sum\limits_{n=1}^\infty a_n < \infty$
	\item $g < 1 \Rightarrow \sum\limits_{n=1}^\infty a_n = \infty$
	\item $g = 1 \Rightarrow $?
\end{itemize}
\end{tw}

\begin{tw}[Kryterium Cauch'ego]~\\
Niech $\forall n\in\mathbb{N} \quad a_n \geqslant 0$ i oznaczamy $g = \limsup\limits_{n\to\infty} \sqrt[n]{a_n}$, wówczas
\begin{itemize}
	\item $g < 1 \Rightarrow \sum\limits_{n=1}^\infty a_n < \infty$
	\item $g < 1 \Rightarrow \sum\limits_{n=1}^\infty a_n = \infty$
	\item $g = 1 \Rightarrow $?
\end{itemize}
\end{tw}

\begin{tw}[Kryterium całkowe zbieżności szeregu]~\\
$f: [1, \infty) \to \mathbb{R}$ funkcja nieujemna i nierosnąca. Wtedy 
$\sum\limits_{n=1}^\infty f(n)$ jest zbieżny $\Leftrightarrow \int\limits_{n=1}^\infty f(x)dx$ jest zbieżna.
\end{tw}

\begin{tw}[Kryterium Dirichleta]~\\
$(a_n)$ to ciąg nierosnący i takiże $\lim\limits_{n\to\infty} a_n = 0$\\
$(b_n)$ to ciąg taki że ciąg sum częściowych jest ograniczony to znaczy 
$\exists M\mathbb{R} ~~ \forall n\in\mathbb{N} \quad |b_1 + b_2 + \dots b_n| \leqslant M$\\
$\Rightarrow \sum\limits_{n=1}^\infty a_nb_n$ jest zbiezny
\end{tw}

\begin{tw}[Kryterium Leibniza]~\\
$(a_n)$ to ciąg nierosnący i taki, że $\lim\limits_{n\to\infty} a_n = 0 \Rightarrow \sum\limits_{n=1}^\infty a_n(-1)^{n+1} = a-1 - a_2 + a_3 - a_4 + \dots$ jest zbieżny
\begin{proof}
Niech $b_n = (-1)^{n+1}$, wówczas $b_1 + b_2 + \dots + b_n = 
	\begin{cases}
		1 \quad n=2k+1\\
		0 \quad n=2k\\
 	\end{cases}
\Rightarrow \forall n\in\mathbb{N} ~~ 0\leqslant b_1 + b_2 + \dots + b_n \leqslant 1$ to znaczy ciąg $(b_1 + b_2 + \dots + b_n)$ jest ograniczony. Są spełnione założenia kryterium Dirichleta $\Rightarrow \sum\limits_{n=1}^\infty a_nb_n = \sum\limits_{n=1}^\infty a_n(-1)^{n+1}$ jest zbiezny. 
\end{proof}
\end{tw}

\begin{df}
Szereg $\sum\limits_{n=1}^\infty a_n$ jest zbieżny bezwzględnie $\Leftrightarrow \sum\limits_{n=1}^\infty |a_n|$ jest zbieżny.
\end{df}

\begin{df}
Szereg który jest zbieżny ale nie jest zbieżny bezwzględnie nazywamy zbieżnym warunkowo.
\end{df}

\begin{tw}
Szereg $\sum\limits_{n=1}^\infty a_n$ jest zbiezny bezwzględnie $\Rightarrow$ Szereg $\sum\limits_{n=1}^\infty a_n$ jest zbieżny i $|\sum\limits_{n=1}^\infty a_n| \leqslant \sum\limits_{n=1}^\infty |a_n|$ 
\end{tw}

%dodać przykłady i wyjaśnieniea




\section{Ciągi i szeregi funkcyjne}
\subsection{Ciągi funkcyjne}
\begin{df}[Punktowej zbieżności]~\\
Mówimy, że ciąg funkcyjny $(f_n)$ jest zbieżny do funkcji $f$ (punktowo)
 $$\Leftrightarrow \forall x\in A \quad \lim\limits_{n\to\infty}\underbrace{f_n(x)}_{ciąg liczbowy} = \underbrace{f(x)}_{liczba} \Leftrightarrow \forall x\in A ~ \forall\varepsilon > 0 ~~ \exists N = N(\varepsilon, x) ~ \forall n \geqslant N \quad |f_n(x)-f(x)| < \varepsilon$$, oznaczamy $f_n \rightarrow f$
\end{df}

\begin{df}[Jednostajnej zbieżności]~\\
Mówimy, że ciąg funkcyjny $(f_n)$ jest zbieżny do funkcji $f$ (jednostajnie) $$\Leftrightarrow \forall\varepsilon > 0 ~~ \exists N = N(\varepsilon, x) ~ \forall n \geqslant N ~ \forall x\in A ~ \quad |f_n(x)-f(x)| < \varepsilon$$, oznaczamy $f_n \rightarrow f$
\end{df}

\begin{tw}[Warunek równoważny zbiezności jednostajnej ciągu funkcyjnego]~\\
$f, f_n: A \to \mathbb{R}$. Wówczas $f_n \rightrightarrows f \Leftrightarrow \lim\limits_{n\to\infty} \sup\limits_{x\in A} |f_n(x)-f(x)| = 0$
\end{tw}

\begin{przyklad}[ciągu który jest zbieżny punktowo ale nie jednostajnie]
$f_n: [0,1] \to \mathbb{R}, \quad f_n(x) = x^n$ zbieżność punktowa
$
	\lim\limits_{n\to\infty}f_n(x) = \lim\limits_{n\to\infty}x^n = 
	\begin{cases}
	0 \quad x\in[0,1)\\
	1 \quad x=1
	\end{cases}
$
\end{przyklad}

\begin{df}[Warunek Cauch'ego dla zbieżności punktowej i jednostajnej]~\\
$\forall x\in A f_n \rightarrow f \Leftrightarrow \forall x\in A$ ciąg $(f_n(x))$ jest zbieżny $\Leftrightarrow \forall x\in A \mbox{ ciąg } (f_n(x))$ spełnia warunek Cauch'ego, to znaczy 
$\forall x\in A ~~ \forall \varepsilon>0 ~~ \exists N ~ \foralln,m \geqslant N \quad |f_n(x) - f_m(x)| < \varepsilon$ 
\end{df}

\begin{tw}
Ciąg funkcyjny $(f_n)$ jest zbieżny jednostajnie na A $\Leftrightarrow$ spełnia warunek Cauch'ego zbiezności jednostajnej.
\end{tw}

\begin{tw}[o ciągłości granicy ciągu funkcyjnego]~\\
$f, f_n: A\to \mathbb{R}, f_n \rightrightarrows_A f$, funkcje $f_n$ są ciągłe w punkcie $a\in A ~~ \forall n\in N \Rightarrow f$ jest ciągła w punkcie $a$
\end{tw}

\begin{tw}[o różniczkowaniu granicy ciągu funkcyjnego]~\\
$A \mbox{ -- przedział } \subset \mathbb{R}$ są różniczkowalne w każdym punkcie przedziału A. Ciąg $(f_n')$ jest z zbieżny jednostajnie na A, czyli $\exists x_0 \in A  \quad (f_n'(x_0))$ jest zbieżny $\Rightarrow$
\begin{itemize}
	\item ciąg $(f_n)$ jest jednostajnie zbieżny na A do pewnej funkcji granicznej
	\item funckja graniczna $f$ jest różniczkowalna na A i $$\forall x\in A ~~ f'(x) = \lim\limits_{n\to\infty} f_n'(x) = (\lim\limits_{n\to\infty} f_n(x))'$$
\end{itemize}

\begin{tw}[o całkowaniu granicy ciągu funkcyjnego]~\\
$A \mbox{ -- przedział } \subset \mathbb{R}, f, f_n \in C(A), f_n \rightrightarrows_A f \Rightarrow \forall a,b \in A \quad \int\limits_a^bf(x)dx = \lim\limits_{n\to\infty}\int\limits_a^bf_n(x)dx$ 
\end{itemize}

\subsection{Szeregi funkcyjne}
% dodać definicje
\begin{tw}
Szereg $\sum\limits_{n=1}^\infty a_n$ jest zbieżny jednostajnie $\Leftrightarrow$ spełnia warunek Cauchego jednostajnej zbieżności szeregu funkcyjnego.
$$\forall \espilon ~~\exists N ~ \forall m>n>N ~~ \forall x\in A \quad |a_{n+1}(x) + \dots + u_m(x)| < \varepsilon $$
\end{tw}

\begin{tw}[Kryterium Weierstrassa]
Jeśli istnieje ciąg liczbowy $(a_n)$ taki, że $\forall n\in \mathbb{N} ~~ \forallx\in A \quad |u_n(x)|\leqslant a_n$ i $\sum\limits_{n=1}^\infty a_n$ jest zbieżny, to szereg funkcyjny $\sum\limits_{n=1}^\infty u_n$ jest zbieżny jednostajny ( i bezwzględnie)
\begin{przyklad}
Szereg $\sum\limits_{n=1}^\infty \frac{sin(nx)}{2^n}$ jest zbieżny jednostajnie na $\mathbb{R}$ bo
$\forall n\in \mathbb{N} ~~ \forall x\in\mathbb{R} \quad \frac{sin(nx)}{2^n} \leqslant \frac{1}{2^n} $ i $\sum\limits_{n=1}^\infty \frac{1}{2^n}$ jest zbieżny jako ciąg geometryczny.
\end{przyklad}
\end{tw}

\begin{tw}[o ciągłości sumy szeregu funkcyjnego]~\\
$S, u_n: A \to \mathbb{R}, ~~ \sum\limits_{n=1}^\infty u_n$ jest zbieżny jednostajnie do funkcji S, funkcje $u_n$ są ciągłe na $A~ \forall n\in\mathbb{N} \Rightarrow S=\sum\limits_{n=1}^\infty u_n$ też jest ciągła na $A$.
\end{tw}

\begin{tw}[o różniczkowaniu sumy szeregu funkcyjnego]~\\
$A$ -- przedział $\subset \mathbb{R}$\\
$u_n: A \to \mathbb{R}$ są różniczkowalne w każdym punkcie przedziału $A$\\
$\sum\limits_{n=1}^\infty u_n'$ jest zbieżny jednostajnie na $A$\\
$\exists x_0\in A ~~ \sum\limits_{n=1}^\infty u_n(x_0)$ jest zbieżby $\Rightarrow$
\begin{itemize}
	\item $\sum\limits_{n=1}^\infty u_n$ jest zbieżny jednostajnie
	\item $\sum\limits_{n=1}^\infty u_n$ jest różniczkowalna na $A$ i $(\sum\limits_{n=1}^\infty u_n)' = \sum\limits_{n=1}^\infty u_n'$
\end{itemize}
\end{tw}

\begin{tw}[o całkowaniu szeregu funkcyjnego]~\\
$A$ -- przedział $\subset \mathbb{R}$\\
$u_n: A \to \mathbb{R}$ są ciągłe na $A$\\
$\sum\limits_{n=1}^\infty u_n$ jest zbieżny jednostajnie\\
$\forall a,b\in A ~~ \int\limits_a^b (\sum\limits_{n=1}^\infty u_n(x))dx = \lim\limits_{n\to\infty}\int\limits_a^b (u_1(x) + u_2(x) + \dots + u_n(x))dx =  \sum\limits_{n=1}^\infty \int\limits_a^b u_n(x)dx$
\end{tw}

\subsection{Szeregi potęgowe}
\begin{df}
Szeregiem potęgowym nazywamy szereg funkcyjny w postaci $\sum\limits_{n=0}^\infty a_nx^n$, gdzie $a_n\in\mathbb{R}$ dla $n\in\mathbb{N}$ 
\end{df}

\begin{tw}
Jesli istnieje granica $\lim\limits_{n\to\infty} |\frac{a_{n+1}}{a_n}| = \lambda $, to promień zbieżności szeregu potęgowego $\sum\limits_{n=0}^\infty a_nx^n$ dany jest wzorem:
$ R = 
	\begin{cases}
		\frac{1}{\lambda}, & \lambda \in (0,\infty)\\
		+\infty, & \lambda = 0\\
		0, & \lambda = \infty
	\end{cases} $
\end{tw}

\begin{tw}[Cauch'ego-Hadamarda]~\\
Promień zbieżności szeregu potęgowego $\sum\limits_{n=0}^\infty a_nx^n$ dany jest wzorem 
$ R = 
	\begin{cases}
		\frac{1}{\lambda}, & \lambda \in (0,\infty)\\
		+\infty, & \lambda = 0\\
		0, & \lambda = \infty
	\end{cases} $, gdzie $\lambda = \limsup\limits_{n\to\infty}\sqrt[n]{|a_n|}$
\end{tw}

\begin{przyklad}[Szeregu o promieniu zbieżności = 7]
$\sum\limits_{n=1}^\infty \frac{1}{n7^n}x^n$
\end{przyklad}
\begin{przyklad}[Szeregu zbieżnego tylko dla x=0]
$\sum\limits_{n=1}^\infty n^nx^n$
\end{przyklad}
\begin{przyklad}[Szeregu zbieżnego tylko dla x=5]
$\sum\limits_{n=1}^\infty n^n(5-x)^n$
\end{przyklad}

\begin{tw}[o ciągłości sumy szeregu potęgowego]~\\
Niech promień zbieżności $R$ szeregu $\sum\limits_{n=0}^\infty a_nx^n$ będzie dodatni. Wówczas funkcja $f(x)=\sum\limits_{n=0}^\infty a_nx^n$ jest ciągła na $(-R, R)$
\end{tw}

\begin{tw}[Abela]
Szereg potęgowy jest funkcją ciągłą w każdym punkcie, w którym jest zbieżny (w punktach końcowych przedziału mówimy o ciągłości jednostronnej) 
\end{tw}

\begin{tw}[o różniczkowaniu szeregu potęgowego]~\\
\begin{itemize}
	\item Oba szeregi $\sum\limits_{n=0}^\infty a_nx^n$ i $\sum\limits_{n=0}^\infty (a_nx^n)'$ mają te same promienie zbieżności
	\item Jeśli $R > 0$, to $f(x) = \sum\limits_{n=0}^\infty a_nx^n$ jest różniczkowalna na $(-R, R)$ i $f'(x) = (\sum\limits_{n=0}^\infty a_nx^n)' = \sum\limits_{n=0}^\infty (a_nx^n)' \quad \forall x\in (-R,R)$
\end{itemize}
\end{tw}

\begin{tw}[o całkowaniu szeregu potęgowego]~\\
\begin{itemize}
	\item Oba szeregi $\sum\limits_{n=0}^\infty a_nx^n$ i $\sum\limits_{n=0}^\infty \int\limits_0^x (a_nt^n)dt$ mają te same promienie zbieżności
	\item Jeśli $R > 0$, to $f(t) = \sum\limits_{n=0}^\infty a_nt^n$ jest całkowalna na $(0, x)$ i $\int\limits_0^x f(t)dt = \int\limits_0^x(\sum\limits_{n=0}^\infty a_nt^n)dt \quad \forall x\in (-R,R)$
\end{itemize}
\end{tw}

\subsection{Szereg Taylora i Maclaurina}
\begin{df}[szeregu Taylora]~\\
Niech $f\in C^\infty ((x_0-\delta, x_0+\delta))$ wtedy szereg potęgowy $\sum\limits_{n=0}^\infty \frac{f^{(n)}(x_0)}{n!}(x-x_0)^n$ nazywamy szeregiem Taylora o środku w punkcie $x_0$ dla funkcji $f$.
\end{df}

\begin{df}[szeregu maclaurina]~\\
Niech $f\in C^\infty ((\delta, \delta))$ wtedy szereg potęgowy $\sum\limits_{n=0}^\infty \frac{f^{(n)}(0)}{n!}x^n$ nazywamy szeregiem Maclaurina dla funkcji $f$.
\end{df}

\begin{przyklad}[Rozwinięć niektórych funkcji w szereg Macalurina]~\\
\begin{itemize}
	\item $\frac{1}{1-x} = \sum\limits_{n=0}^\infty x^n \quad |x| < 1$
	\item $e^x = \sum\limits_{n=0}^\infty \frac{x^n}{n!}$
	\item $sinx = \sum\limits_{n=0}^\infty \frac{(-1)^n}{(2n+1)!}x^{2n+1}$
	\item $cosx = \sum\limits_{n=0}^\infty \frac{(-1)^n}{(2n)!}x^{2n}$
\end{itemize}

\section{Przestrzenie metryczne i unormowane}
\begin{df}
Przestrzenią metryczną nazywamy parę $(X, \varrho)$, gdzie $X$ to niepusty zbiór, a $\varrho$ to metryka w tym zbiorze. Elementy $X$ nazywamy punktami zaś $\varrho (x,y)$ odległościa między $x$ i $y$ 
\end{df}
\begin{przyklad}[Metryka naturalna (euklidesowa)]
$X = \mathbb{R}^2 ~\varrho((x_1,x_2),(y_1, y_2)) = \sqrt{(x_1-y_1)^2+(x_2-y_2)^2}$
\end{przyklad}
\begin{przyklad}[Metryka dyskretna]
$X \mbox{-- dowolny zbiór niepusty} \quad \varrho(x,y) = 
\begin{cases}
0 \mbox{ dla } x = y\\
1 \mbox{ dla } x \neq y
\end{cases}$
\end{przyklad}
\begin{przyklad}[Metryka taksówkowa (miejska)]
$X = \mathbb{R}^n ~\varrho(x, y) = \sum\limits_{k=1}^n |x_k-y_k|$
\end{przyklad}

\begin{df}
Kulą (otwartą) o środku w punkcie $x_0$ i promieniu $r$ w przestrzeni metrycznej $(X, \varrho)$ nazywamy zbiór $K(x_0, r) = \{x\in X: \varrho(x,x_0) < r\}$
\end{df}
\begin{df}
Kulą domkniętą o środku w punkcie $x_0$ i promieniu $r$ w przestrzeni metrycznej $(X, \varrho)$ nazywamy zbiór $\overline{K}(x_0, r) = \{x\in X: \varrho(x,x_0) \leqslant r\}$
\end{df}
\begin{df}
Sferą o środku w punkcie $x_0$ i promieniu $r$ w przestrzeni metrycznej $(X, \varrho)$ nazywamy zbiór $S(x_0, r) = \{x\in X: \varrho(x,x_0) = r\}$
\end{df}

\begin{przyklad}[Kula w metryce dyskretnej]
$$K(x_0, r) = \{x\in X: \varrho(x,x_0) < r\} = \begin{cases}
	\{x_0\} &\mbox{ gdy } r\in (0,1]\\
	X &\mbox{ gdy } r\in (1, \infty)
\end{cases}$$
\end{przyklad}
\begin{przyklad}[Kula domknięta w metryce dyskretnej]
$$\overline{K}(x_0, r) = \{x\in X: \varrho(x,x_0) \leqslant r\} = \begin{cases}
	\{x_0\} &\mbox{ gdy } r\in (0,1)\\
	X &\mbox{ gdy } r\in [1, \infty)
\end{cases}$$
\end{przyklad}
\begin{przyklad}[Sfera w metryce dyskretnej]
$$S(x_0, r) = \{x\in X: \varrho(x,x_0) = r\} = \begin{cases}
	\emptyset &\mbox{ gdy } r\in (0,1) \cup (1,\infty)\\
	X\smallsetminus\{x_0\} &\mbox{ gdy } r=1
\end{cases}$$
\end{przyklad}

\begin{df}
Ciąg $(a_n)$ o wyrazach w przestrzeni metrycznej $(X, \varrho)$ jest zbiezny do $a\in X \Leftrightarrow \forall\varepsilon >0~~ \exists N ~ \forall n \geqslant N \quad \varrho(a_n, a) < \varepsilon \Leftrightarrow \forall\varepsilon >0~~ \exists N ~ \forall n \geqslant N \quad |\varrho(a_n, a) - 0| < \varepsilon \Leftrightarrow \varrho(a_n, a) \rightarrow 0$
\end{df}

\begin{df}
Ciąg $(a_n)$ o wyrazach w przestrzeni metrycznej spelnia warunek Cauch'ego $\Leftrightarrow \forall \varepsilon >0 ~~ \exists N ~ \foralln,m > N \quad \varrho(a_n, a_m) < \varepsilon$
\end{df}

\begin{tw}
Ciąg $(a_n)$ o wyrazach w przestrzeni metrycznej jest zbieżny $\Rightarrow$ spełnia warunek Cauch'ego
\end{tw}

\begin{przyklad}
Ciąg $a_n = \frac{1}{n}$ o wyrazach w przestrzeni metrycznej $X = (0, \infty)$ z metryką naturalną spełnia warunek Cauch'ego a mimo to nie jest zbieżny bo jedyny kandydat na granicę -- $0$ odpada
\end{przyklad}

\begin{df}
Przestrzeń metryczna, w której każdy ciąg spełniający warunek Cauch'ego jest zbieżny nazywamy przestrzenią metryczną zupełną.
\end{df}

\begin{tw}
Przestrzeń $X = \mathbb{R}$ i $\varrho (x,y) = |x-y|$ jest przestrzenią zupełną
\end{tw}

\begin{tw}[Banach o punkcie stałym]~\\
Niech $(X, \varrho)$ będzie przestrzenią metryczną zupełną i $f: X \to X$ będzie odwzorowaniem zwężającym, to znaczy funkcją spełniająca warunek
$$ \exists L\in (0,1) ~~ \forall x,y\in X \quad \varrho (f(x), f(y)) \leqslant L\varrho (x, y) $$
Wówczas istnieje dokładnie jedno $x_0 \in X$ (nazywane punktem stałym odwzorowania), takie że $f(x_0) = x_0$. 
\end{tw}


\subsection{Elementy topologii}
\begin{df}
Zbiór $A$ zawarty w przestrzeni metrycznej $(X, \varrho)$ nazywamy otwartym jesli każdy punkt $a$ ze zbioru $A$ nalezy do $A$ wraz z pewną kulą o środku w $a$.\\
$$A \mbox{ jest otwarty } \Leftrightarrow \forall a\in A ~~ \exists r>0 \quad K(a, r) \subset A$$
\end{df}
\begin{df}
Zbiór $A$ zawarty w przestrzeni metrycznej $(X, \varrho)$ nazywamy domkniętym $ \Leftrightarrow X \smallsetminus A$ jest otwarty
\end{df}

\begin{df}
Wnętrze zbioru $A$ w przestrzenimetrycznej $(X, \varrho)$ to 
$$ IntA := \{a\in A: ~~ \exists r>0 ~~ K(a,r) \subset A \} $$
\end{df}
\begin{df}
Domknięcie zbioru $A$ w przestrzenimetrycznej $(X, \varrho)$ to 
$$ \overline{A} := \{x\in X: ~~ \exists r>0 ~~ K(x,r) \cap A \neq \emptyset \} $$
\end{df}

\begin{przyklad}
Rozpatrzmyu przestrzen metryczną $(X, \varrho)$, gdzie $X = \mathbb{R}$, $\varrho (x,y) = |x-y|$\\
$$Int([a,b)) = (a,b) \quad\quad \overline{(a,b)} = [a,b]$$
\end{przyklad}

\end{document} 

