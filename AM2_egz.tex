\documentclass[12pt,a4paper]{article}
\usepackage[centertags]{amsmath}
\usepackage[polish]{babel}
\usepackage[utf8]{inputenc}
\usepackage[T1]{fontenc}
\usepackage[a4paper]{geometry}
\usepackage{amsfonts}
\usepackage{fullpage}
%\usepackage{epsfig}
\usepackage{amsthm}
\usepackage{pdflscape}
\usepackage{float}
%\usepackage{newlfont}
\usepackage{amsmath} %pakiet potrzebny do \eqref
\usepackage{amssymb} %pakiet potrzebny do \blacksquare i \mathbb{•}

\newtheorem{lemat}{Lemat}
\newtheorem{tw}{Twierdzenie}
\newtheorem{przyklad}{Przykład}
\newtheorem{wn}{Wniosek}
\newtheorem{zad}{Zadanie}
\theoremstyle{definition}
\newtheorem{df}{Definicja}
\usepackage[polish]{babel}
\usepackage[utf8]{inputenc}



\pagestyle{plain}
\begin{document}
\title{ Analiza matematyczna 2}
\author{Tomasz Janiszewski}
\date{\today}
\maketitle
\thispagestyle{empty}

\section{Całka Riemanna}
\begin{df}[Ciąg normalny podziałów]
Ciąg podziałów $(\pi(n)$ nazywamy normalnym, jeśli $\delta(\pi(n))\xrightarrow{n\to\infty}0$
\end{def}
\begin{df}[Sumy całkowe]~\\
\begin{array}{l l}
\mbox{Dolna suma całkowa:}& $s_n = s_n(\pi_n) = \sum\limits_{i=1}^{k_n}m_i^{(n)}(x_i^{(n)}-x_{i-1}^{(n)}})$\\
\mbox{Górna suma całkowa:}& $S_n = S_n(\pi_n) = \sum\limits_{i=1}^{k_n}M_i^{(n)}(x_i^{(n)}-x_{i-1}^{(n)}})$\\
\mbox{Suma całkowa Riemanna:}& $\sigma_n = \sigma_n(\pi_n) = \sum\limits_{i=1}^{k_n}f(\xi_i^{(n)})(x_i^{(n)}-x_{i-1}^{(n)}})$
\end{array}
\end{df}
\begin{df}[Funkcja całkowalna w sensie Riemanna]
Funkcja f jest całkowalna w sensie Riemanna na $[a,b] \Leftrightarrow$ instnieje $\sigma\in\mathbb{R}$ taka,że dla dowolnego normalnego ciągu podziałów $(\pi_n)$ oraz dla dowolnego wartościowania tego ciągu $(\omega_n)$ mamy $\lim\limits_{n\to\infty}\sigma_n(\pi_n,\omega_0) = \lim\limits_{n\to\infty}\sum\limits_{i=1}^{k_n}f(\xi_i^{(n)})(x_i^{(n)}-x_{i-1}^{(n)}})$, to $\sigma$ nazywamy całką Riemanna funkcji f na $<a,b>$ i oznaczamy $\int\limits_a^b(f(x)dx$
\end{df}
\begin{tw}
$f: [a,b] \rightarrow \mathbb{R}$ (ograniczona) jest całkowalna w sensie Riemanna na $[a,b] \Leftrightarrow s=S$
\end{tw}
\begin{tw}
Każda funkcja monoticzna i ograniczona na $[a,b]$ jest całkowalna w sensie Riemanna na $[a,b]$
\end{tw}
\begin{tw}
Każda funkcja ciągła na $[a,b]$ jest całkowalna w sensie Riemanna na $[a,b]$
\end{tw}

\subsection{Własności całki Riemanna}

\begin{enumerate}
\item $f,g: [a,b] \to \mathbb{R} {x\in[a,b]: f(x) \neq g(x)}$ jest zbiorem skończonym $\Rightarrow$ f jest całkowalna na $[a,b]\Leftrightarrow$ g jest całkowalna na $[a,b]$. W przypadku całkowalności $\int\limits_a^bf(x)dx=\int\limits_a^bg(x)dx$
\item Liniowość
\item $f: [a,b]\to\mathbb{R}$ całkowalna na $[a,b] \Rightarrow$ f jest całkowalna na kazdym podprzedziale $[a,b]$
\item $f: [a,b]\to\mathbb{R}$, f jest całkowalna na $[a,c]$ oraz $[c,b] \Rightarrow$ f jest całkowalna na $[a,b]$ oraz $\int\limits_a^bf(x)dx = \int\limits_a^cf(x)dx + \int\limits_c^bf(x)dx$
\item $f: [a,b] \to \mathbb{R}$ całkowalna na $[a,b]$, $\forall x\in[a,b] f(x) \geq 0 \Rightarrow \int\limits_a^bf(x)dx \geq 0$
\item $f,g: [a,b] \to \mathbb{R}$ całkowalne na $[a,b]$, $\forall x\in[a,b] f(x) \leq g(x) \Rightarrow \int\limits_a^bf(x)dx \leq \int\limits_a^bg(x)dx$
\item $f: [a,b] \to \mathbb{R}$ całkowalna na $[a,b] \Rightarrow$ $|f|$ też jest całkowalna na $[a,b]$ oraz $|\int\limits_a^bf(x)dx| \leq \int\limits_a^b|f(x)|dx$
\item $f: [a,b] \to \mathbb{R}$ całkowalna na $[a,b] \Rightarrow \int\limits_a^bf(x)dx \leq \sup\limits_{x\in[a,b]}f(x)\cdot(b-a)$
\item Podstawowy wzór rachunku całkowego\\
$f: [a,b] \to \mathbb{R}$ funkcja ciągła, $\phi(x)$ --- dowolna funkcja pierwotna dla $f(x)$, to znaczy $\phi'(x)=f(x) \Rightarrow \int\limits_a^bf(x)dx = \phi(b)-\phi(a)$
\item Całkowanie przez podstawienie\\
$
f: [a,b]\to\mathbb{R}$, funkcja ciągła, $g: [a,b]\to\mathbb{R},~ g\in C^1([a,b])\\
\alpha = g(a),~ \beta =g(b)\\
\int\limits_a^bf(g(x))g'(x)dx = \int\limits_\alpha^\beta f(t)dt, \mbox{ gdzie }t = g(x)
$
\begin{proof}~\\
	Niech $\phi(x)$ będzie funkcją pierwotną dla $f(x)$, to znaczy
	 $\phi'(x) = f(x), \phi=\int\limits_a^bf(t)dt = \phi(\beta)-\phi(\alpha)$\\
$[\phi(g(x))]' = \phi'(g(x))g'(x) = f(g(x))g'(x) \Rightarrow \phi(g(x)) \mbox{ to funkcja pierwotna funkcji } f(g(x))g'(x)$\\
$L = \int\limits_a^bf(g(x))g'(x)dx = \phi(g(b))-\phi(g(a)) = \phi(\beta)-\phi(\alpha)\\
L = P$
\end{proof}
\item Całkowanie przez cześci\\
$
u,v: [a,b]\to\mathbb{R} \mbox{ --- funkcje klasy } C'\\
\int\limits_a^bu(x)v'(x)dx = [u(x)v(x)]\limits_a^b - \int\limits_a^bu'(x)v(x)dx
$
\begin{proof}~\\
$
	[u(x)v(x)]' = u'(x)v(x)+u(x)v'(x)\\
	\int\limits_a^b[u(x)v(x)]'dx = \int\limits_a^b(u'(x)v(x)+u(x)v'(x))dx = 
	\int\limits_a^b u'(x)v(x)dx + \int\limits_a^bu(x)v'(x)dx \quad(*)\\
	u(x)v(x) \mbox{ to funkcja pierwotna } [u(x)v(x)]'\\
	\int\limits_a^b[u(x)v(x)]'dx u(b)v(b) - u(a)v(a) = [u(x)v(x)]\limits_a^b \quad(**)\\
	(*):(**) \Rightarrow [u(x)v(x)]\limits_a^b = \int\limits_a^bu'(x)v(x)dx + \int\limits_a^b u(x)v'(x)dx\\
	\int\limits_a^bu(x)v(x)dx = [u(x)v(x)]_a^b-\int\limits_a^bu'(x)v(x)dx	
$
\end{proof}
\item Twierdzenie o wartości średniej rachunku całkowego
$
f,g: [a,b]\to\mathbb{R} \mbox{ --- funkcje ciągłe, g jest nieujemna (niedodatnia)}\\
 \exists \xi\in [a,b] \int\limits_a^bf(x)g(x)dx = f(\xi)\int\limits_a^bg(x)dx
$
\end{enumerate}

\subsection{Całki niewłaściwe}

\begin{df}
$f: [a, \infty) \to \mathbb{R} \mbox{ całkowalna na } [\alpha, \beta] \forall \beta > \alpha$ oraz istnieje granica $\lim\limits_{\beta\to\infty} \int\limits_\alpha^\beta f(x)dx \Rightarrow$ granicę tę nazywamy całka niewłaściwą pierwszego rodzaju i oznaczamy $\int\limits_\alpha^\beta f(x)dx := \lim\limits_{\beta\to\infty} \int\limits_\alpha^\beta f(x)dx$\\
Ponadto jeśli granica ta istnieje i jest skończona to całkę niewłaściwą nazywamy zbieżną. Natomiast w pozostałych przypadkach całkę niewłaściwą nazywamy rozbieżną.
\end{df}

\begin{df}
$f: [a, \infty) \to \mathbb{R} \mbox{ całkowalna na } [\alpha, \beta] \quad \forall a < \beta < b$ oraz istnieje granica $\lim\limits_{\beta\to b^-} \int\limits_a^\beta f(x)dx \Rightarrow$ granicę tę nazywamy całka niewłaściwą drugiego rodzaju i oznaczamy $\int\limits_a^b f(x)dx := \lim\limits_{\beta\to b^-} \int\limits_\alpha^\beta f(x)dx$\\
Ponadto jeśli granica ta istnieje i jest skończona to całkę niewłaściwą nazywamy zbieżną. Natomiast w pozostałych przypadkach całkę niewłaściwą nazywamy rozbieżną.
\end{df}

\subsection{Kryteria zbieżności}

\begin{tw}[Kryterium porównawcze]
$f,g: [a,b) \to \mathbb{R} \mbox{ całkowalna na } [\alpha, \beta] \quad \forall a < \beta < b$ oraz
$\forall x\in [a,b) \quad 0<leq f(x) \leq g(x)$ Wtedy:
\begin{itemize}
\item $\int\limits_a^bg(x)dx$ jest zbieżna $\Rightarrow \int\limits_a^bg(x)dx$ też jest zbieżna
\item $\int\limits_a^bf(x)dx$ jest rozbieżna $\Rightarrow \int\limits_a^bg(x)dx$ też jest rozbieżna
\end{itemize}
\end{tw}

\begin{tw}
$f: [a,b) \to \mathbb{R} \mbox{ całkowalna na } [\alpha, \beta] \quad \forall a < \beta < b$ oraz
$\int\limits_a^b \f(x)|dx$ jest zbieżna, to $\int\limits_a^b f(x)dx$ też jest zbieżna. Mówimy wtedy, że jest zbieżna bezwzględnie.
\end{tw} 

\subsection{Zastosowania geometryczne całki Riemanna}

\begin{tw}[Pole zbioru płaskiego]
$f: [a,b] \to \mathbb{R}$ funkcja nieujemna i ciągła, $D = \{(x,y) \in \mathbb{R}^2: x\in [a,b[, y\in [0, f(x)]$\\
pole $D = |D| = \int\limits_a^b f(x)dx$
\end{tw}

\begin{tw}[Długość łuku]
Niech $f: [a,b] \to \mathbb{R}$ - funkcja klasy $C^1$. Wówczs długość łuku opisanego równaniem $y=f(x), x\in [a,b]$ dana jest wzorem $L = \int\limits_a^b \sqrt{1 + \left[f'(x)\right]^2}$
\end{tw}

\begin{tw}[Objętość bryły]
Niech $f: [a,b] \to \mathbb{R}$ - funkcja klasy $C^1$ oraz $V$ oznacza bryłę powstałą poprzez obrót dookoła osi OX krzywej $y - f(x), x\in [a,b]$. Wówczs objętość $V$ dana jest wzorem $V = \Pi\int\limits_a^b f^2(x)dx
\end{tw}


\end{document} 

