\documentclass{article}
\usepackage{amsfonts}
\usepackage{amssymb}
\usepackage[centertags]{amsmath}
\usepackage{amsthm}
\usepackage{newlfont}
\newtheorem{lemat}{Lemat}
\newtheorem{tw}{Twierdzenie}
\newtheorem{przyklad}{Przykład}
\newtheorem{wn}{Wniosek}
\newtheorem{zad}{Zadanie}
\theoremstyle{definition}
\newtheorem{df}{Definicja}
\usepackage[polish]{babel}
\usepackage[utf8]{inputenc}
\textwidth 18cm
\textheight 24.5cm
\topmargin -1cm
\oddsidemargin 0cm
\evensidemargin 0cm
\def\thefootnote{\arabic{footnote})}


\pagestyle{plain}
\begin{document}
\title{ Analiza matematyczna 2}
\author{Tomasz Janiszewski}
\date{\today}
\maketitle
\thispagestyle{empty}

\section{Całka Riemanna}
\begin{df}[Ciąg normalny podziałów]
Ciągie podziałów $(\pi(n)$ nazywamy normalnym, jeśli $\delta(\pi(n))\xrightarrow{n\to\infty}0$
\end{def}
\begin{df}[Sumy całkowe]~\\
\begin{array}{l l}
\mbox{Dolna suma całkowa:}& $s_n = s_n(\pi_n) = \sum\limits_{i=1}^{k_n}m_i^{(n)}(x_i^{(n)}-x_{i-1}^{(n)}})$\\
\mbox{Górna suma całkowa:}& $S_n = S_n(\pi_n) = \sum\limits_{i=1}^{k_n}M_i^{(n)}(x_i^{(n)}-x_{i-1}^{(n)}})$\\
\mbox{Suma całkowa Riemanna:}& $\sigma_n = \sigma_n(\pi_n) = \sum\limits_{i=1}^{k_n}f(\xi_i^{(n)})(x_i^{(n)}-x_{i-1}^{(n)}})$
\end{array}
\end{df}
\begin{df}[Funkcja całkowalna w sensie Riemanna]
Funkcja f jest całkowalna w sensie Riemanna na $<a,b> \Leftrightarrow$ instnieje $\sigma\in\mathbb{R}$ taka,że dla dowolnego normalnego ciągu podziałów $(\pi_n)$ oraz dla dowolnego wartościowania tego ciągu $(\omega_n)$ mamy $\lim\limits_{n\to\infty}\sigma_n(\pi_n,\omega_0) = \lim\limits_{n\to\infty}\sum\limits_{i=1}^{k_n}f(\xi_i^{(n)})(x_i^{(n)}-x_{i-1}^{(n)}})$, to $\sigma$ nazywamy całką Riemanna funkcji f na $<a,b>$ i oznaczamy $\int\limits_a^b(f(x)dx$
\end{df}

\end{document} 
