\section{Szeregi liczbowe}
\begin{df}[sumy szeregu liczbowego]~\\
Jeśli istnieje skończona lub nie $\lim\limits_{n\to\infty}S_n$, to nazywamy ją sumą szeregu
$\sum\limits_{n=1}^\infty a_n$ i zapisujemy $\sum\limits_{n=1}^\infty a_n=\lim\limits_{n\to\infty}S_n$.
Jeśli $\lim\limits_{n\to\infty}S_n$ jest skończona, to szereg $\sum\limits_{n=1}^\infty a_n$ nazywamy zbieżnym; w
pozostałych przypadkach (to znaczy gdy granica jest nieskończona lub nie istnieje) szereg ten nazywamy rozbieżnym
\end{df}

\begin{tw}[Warunek konieczny zbieżności szeregu]~\\
$\sum\limits_{n=1}^\infty a_n$ jest zbieżny $\Rightarrow \lim\limits_{n\to\infty}a_n = 0$
\begin{proof}
Zakładamy, że $\sum\limits_{n=1}^\infty a_n$ jest zbieżny $\Rightarrow (S_n)$ jest zbieżny; oznaczamy
$S_n \xrightarrow{n\to\infty} S$\\
$a_n = S_n - S_{n-1} \xrightarrow{n\to\infty} S - S = 0$
\end{proof}
\begin{przyklad}[Powyższy warunek nie jest warunkiem dostatecznym]~\\
$\sum\limits_{n=1}^\infty \frac{1}{n}$ nie jest zbieżny, mimo że $\lim\limits_{n\to\infty} \frac{1}{n} = 0$
\end{przyklad}
\end{tw}

\begin{tw}[Warunek Cauch'ego zbieżności szeregu]~\\
Szereg $\sum\limits_{n=1}^\infty a_n$ jest zbieżny $\Leftrightarrow$ spełnia warunek Cauch'ego, to znaczy
$$
\forall \varepsilon > 0 ~~\exists N ~~\forall m > n > N \quad |a_{n+1} + a_{n+2} + \dots + a_m| < \varepsilon
$$
\end{tw}

\begin{tw}[O mnożeniu szeregu przez stałą]~\\
$\sum\limits_{n=1}^\infty a_n$ jest zbieżny i $\lambda\in\mathbb{R}$ wówczas szereg $\sum\limits_{n=1}^\infty \lambda
a_n$ jest zbieżny i $\sum\limits_{n=1}^\infty \lambda a_n = \lambda \sum\limits_{n=1}^\infty a_n$
\end{tw}
\begin{tw}[O dodawaniu i odejmowaniu szeregów]~\\
$\sum\limits_{n=1}^\infty a_n, \sum\limits_{n=1}^\infty b_n$ są zbieżne wówczas szereg $\sum\limits_{n=1}^\infty a_n
\pm \sum\limits_{n=1}^\infty b_n$ jest zbieżny i $\sum\limits_{n=1}^\infty a_n \pm \sum\limits_{n=1}^\infty b_n =
\sum\limits_{n=1}^\infty (a_n \pm b_n)$
\end{tw}

\subsection{Kryteria zbieżności szeregów}

\begin{tw}[Kryterium porównawcze]~\\
$\forall n\in\mathbb{N} \quad 0 \leqslant a_n \leqslant b_n$, wówczas
\begin{itemize}
	\item $\sum\limits_{n=1}^\infty b_n$ jest zbieżny $\Rightarrow \sum\limits_{n=1}^\infty a_n$ jest zbieżny
	\item $\sum\limits_{n=1}^\infty a_n$ jest rozbieżny $\Rightarrow \sum\limits_{n=1}^\infty b_n$ jest rozbieżny
\end{itemize}
\end{tw}

\begin{tw}[Kryterium d'Alemberta]~\\
Niech $\forall n\in\mathbb{N} \quad a_n > 0$ i istnieje granica
$\lim\limits_{n\to\infty} \frac{a_{n+1}}{a_n} = g$, wówczas
\begin{itemize}
	\item $g < 1 \Rightarrow \sum\limits_{n=1}^\infty a_n < \infty$
	\item $g > 1 \Rightarrow \sum\limits_{n=1}^\infty a_n = \infty$
	\item $g = 1 \Rightarrow $?
\end{itemize}
\end{tw}

\begin{tw}[Kryterium Cauch'ego]~\\
Niech $\forall n\in\mathbb{N} \quad a_n \geqslant 0$ i oznaczamy $g = \limsup\limits_{n\to\infty} \sqrt[n]{a_n}$,
wówczas
\begin{itemize}
	\item $g < 1 \Rightarrow \sum\limits_{n=1}^\infty a_n < \infty$
	\item $g > 1 \Rightarrow \sum\limits_{n=1}^\infty a_n = \infty$
	\item $g = 1 \Rightarrow $?
\end{itemize}
\end{tw}

\begin{tw}[Kryterium całkowe zbieżności szeregu]~\\
$f: [1, \infty) \to \mathbb{R}$ funkcja nieujemna i nierosnąca. Wtedy
$\sum\limits_{n=1}^\infty f(n)$ jest zbieżny $\Leftrightarrow \int\limits_{n=1}^\infty f(x)dx$ jest zbieżna.
\end{tw}

\begin{tw}[Kryterium Dirichleta]~\\

$\left.
\begin{array}{l}
	(a_n) \text{ to ciąg nierosnący i taki że } \lim\limits_{n\to\infty} a_n = 0\\
	(b_n) \text{ to ciąg taki że ciąg sum częściowych jest ograniczony }\footnotemark[1]
\end{array}
\right\rbrace \Rightarrow \sum\limits_{n=1}^\infty a_nb_n$ jest zbieżny
\footnotetext[1]{ to znaczy $\exists M\in\mathbb{R} ~~ \forall n\in\mathbb{N} \quad |b_1 + b_2 + \dots b_n| \leqslant
M$}
\end{tw}

\begin{tw}[Kryterium Leibniza]~\\
$(a_n)$ to ciąg nierosnący i taki, że $\lim\limits_{n\to\infty} a_n = 0 \Rightarrow \sum\limits_{n=1}^\infty
a_n(-1)^{n+1} = a-1 - a_2 + a_3 - a_4 + \dots$ jest zbieżny
\begin{proof}
Niech $b_n = (-1)^{n+1}$, wówczas $b_1 + b_2 + \dots + b_n =
	\begin{cases}
		1 \quad n=2k+1\\
		0 \quad n=2k\\
 	\end{cases}
\Rightarrow \forall n\in\mathbb{N} ~~ 0\leqslant b_1 + b_2 + \dots + b_n \leqslant 1$ to znaczy ciąg $(b_1 + b_2 +
\dots + b_n)$ jest ograniczony. Są spełnione założenia kryterium Dirichleta $\Rightarrow \sum\limits_{n=1}^\infty
a_nb_n = \sum\limits_{n=1}^\infty a_n(-1)^{n+1}$ jest zbieżny.
\end{proof}
\end{tw}

\begin{df}
Szereg $\sum\limits_{n=1}^\infty a_n$ jest zbieżny bezwzględnie $\Leftrightarrow \sum\limits_{n=1}^\infty |a_n|$ jest
zbieżny.
\end{df}

\begin{df}
Szereg który jest zbieżny ale nie jest zbieżny bezwzględnie nazywamy zbieżnym warunkowo.
\end{df}

\begin{tw}
Szereg $\sum\limits_{n=1}^\infty a_n$ jest zbieżny bezwzględnie $\Rightarrow$ Szereg $\sum\limits_{n=1}^\infty a_n$
jest zbieżny i $|\sum\limits_{n=1}^\infty a_n| \leqslant \sum\limits_{n=1}^\infty |a_n|$
\end{tw}

%dodać przykłady i wyjaśnieniea

