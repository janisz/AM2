\section{Przestrzenie metryczne i unormowane}
\begin{df}
Przestrzenią metryczną nazywamy parę $(X, \rho)$, gdzie $X$ to niepusty zbiór, a $\rho$ to metryka w tym zbiorze.
Elementy $X$ nazywamy punktami zaś $\rho (x,y)$ odległością między $x$ i $y$
\end{df}
\begin{przyklad}[Metryka naturalna (euklidesowa)]
$X = \mathbb{R}^2 ~\rho((x_1,x_2),(y_1, y_2)) = \sqrt{(x_1-y_1)^2+(x_2-y_2)^2}$
\end{przyklad}
\begin{przyklad}[Metryka dyskretna]
$X \text{-- dowolny zbiór niepusty} \quad \rho(x,y) =
\begin{cases}
0 \text{ dla } x = y\\
1 \text{ dla } x \neq y
\end{cases}$
\end{przyklad}
\begin{przyklad}[Metryka taksówkowa (miejska)]
$X = \mathbb{R}^n ~\rho(x, y) = \sum\limits_{k=1}^n |x_k-y_k|$
\end{przyklad}

\begin{df}
Kulą (otwartą) o środku w punkcie $x_0$ i promieniu $r$ w przestrzeni metrycznej $(X, \rho)$ nazywamy zbiór
$$K(x_0, r) = \{x\in X: \rho(x,x_0) < r\}$$
\end{df}
\begin{df}
Kulą domkniętą o środku w punkcie $x_0$ i promieniu $r$ w przestrzeni metrycznej $(X, \rho)$ nazywamy zbiór
$$\overline{K}(x_0, r) = \{x\in X: \rho(x,x_0) \leqslant r\}$$
\end{df}
\begin{df}
Sferą o środku w punkcie $x_0$ i promieniu $r$ w przestrzeni metrycznej $(X, \rho)$ nazywamy zbiór $$S(x_0, r) =
\{x\in X: \rho(x,x_0) = r\}$$
\end{df}

\begin{przyklad}[Kula w metryce dyskretnej]
$$K(x_0, r) = \{x\in X: \rho(x,x_0) < r\} = \begin{cases}
	\{x_0\} &\text{ gdy } r\in (0,1]\\
	X &\text{ gdy } r\in (1, \infty)
\end{cases}$$
\end{przyklad}
\begin{przyklad}[Kula domknięta w metryce dyskretnej]
$$\overline{K}(x_0, r) = \{x\in X: \rho(x,x_0) \leqslant r\} = \begin{cases}
	\{x_0\} &\text{ gdy } r\in (0,1)\\
	X &\text{ gdy } r\in [1, \infty)
\end{cases}$$
\end{przyklad}
\begin{przyklad}[Sfera w metryce dyskretnej]
$$S(x_0, r) = \{x\in X: \rho(x,x_0) = r\} = \begin{cases}
	\emptyset &\text{ gdy } r\in (0,1) \cup (1,\infty)\\
	X\smallsetminus\{x_0\} &\text{ gdy } r=1
\end{cases}$$
\end{przyklad}

\begin{df}[zbieżności ciągu o wyrazach w przestrzeni metrycznej]~\\
Ciąg $(a_n)$ o wyrazach w przestrzeni metrycznej $(X, \rho)$ jest zbieżny do $a\in X \Leftrightarrow
\forall\varepsilon >0~~ \exists N ~ \forall n \geqslant N \quad \rho(a_n, a) < \varepsilon \Leftrightarrow
\forall\varepsilon >0~~ \exists N ~ \forall n \geqslant N \quad |\rho(a_n, a) - 0| < \varepsilon \Leftrightarrow
\rho(a_n, a) \rightarrow 0$
\end{df}

\begin{df}
Ciąg $(a_n)$ o wyrazach w przestrzeni metrycznej spełnia warunek Cauch'ego $\Leftrightarrow \forall \varepsilon >0 ~~
\exists N ~ \forall n,m > N \quad \rho(a_n, a_m) < \varepsilon$
\end{df}

\begin{tw}
Ciąg $(a_n)$ o wyrazach w przestrzeni metrycznej jest zbieżny $\Rightarrow$ spełnia warunek Cauch'ego
\end{tw}

\begin{przyklad}
Ciąg $a_n = \frac{1}{n}$ o wyrazach w przestrzeni metrycznej $X = (0, \infty)$ z metryką naturalną spełnia warunek
Cauch'ego a mimo to nie jest zbieżny bo jedyny kandydat na granicę -- $0$ odpada
\end{przyklad}

\begin{df}[Przestrzeni metrycznej zupełnej]
Przestrzeń metryczna, w której każdy ciąg spełniający warunek Cauch'ego jest zbieżny nazywamy przestrzenią
metryczną zupełną.
\end{df}

\begin{tw}
Przestrzeń $X = \mathbb{R}$ i $\rho (x,y) = |x-y|$ jest przestrzenią zupełną
\end{tw}

\begin{tw}[Banach o punkcie stałym]~\\
Niech $(X, \rho)$ będzie przestrzenią metryczną zupełną i $f: X \to X$ będzie odwzorowaniem zwężającym, to
znaczy funkcją spełniająca warunek
$$ \exists L\in (0,1) ~~ \forall x,y\in X \quad \rho (f(x), f(y)) \leqslant L\rho (x, y) $$
Wówczas istnieje dokładnie jedno $x_0 \in X$ (nazywane punktem stałym odwzorowania), takie że $f(x_0) = x_0$.
\end{tw}


\subsection{Elementy topologii}
\begin{df}[Zbioru otwartego]~\\
Zbiór $A$ zawarty w przestrzeni metrycznej $(X, \rho)$ nazywamy otwartym jeśli każdy punkt $a$ ze zbioru $A$ należy
do $A$ wraz z pewną kulą o środku w $a$.\\
$$A \text{ jest otwarty } \Leftrightarrow \forall a\in A ~~ \exists r>0 \quad K(a, r) \subset A$$
\end{df}
\begin{df}[Zbioru domkniętego]~\\
Zbiór $A$ zawarty w przestrzeni metrycznej $(X, \rho)$ nazywamy domkniętym $ \Leftrightarrow X \smallsetminus A$ jest
otwarty
\end{df}

\begin{df}
Wnętrze zbioru $A$ w przestrzeni metrycznej $(X, \rho)$ to
$$ IntA := \{a\in A: ~~ \exists r>0 ~~ K(a,r) \subset A \} $$
\end{df}
\begin{df}
Domknięcie zbioru $A$ w przestrzeni metrycznej $(X, \rho)$ to
$$ \overline{A} := \{x\in X: ~~ \exists r>0 ~~ K(x,r) \cap A \neq \emptyset \} $$
\end{df}

\begin{przyklad}
Rozpatrzmy przestrzeń metryczną $(X, \rho)$, gdzie $X = \mathbb{R}$, $\rho (x,y) = |x-y|$\\
$$Int([a,b)) = (a,b) \quad\quad \overline{(a,b)} = [a,b]$$
\end{przyklad}